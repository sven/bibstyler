\documentclass[a4paper,12pt]{scrartcl}
\usepackage[utf8]{inputenc}
\usepackage[T1]{fontenc}
\usepackage{hyperref}
\usepackage{csquotes}

% BibTex
\usepackage[%
    bibencoding=utf8,
    bibstyle=tmp/main,
    firstinits=true,
    labelnumber=true,
]{biblatex}
\newcommand\lit[1]{Citation \cite{#1}}
\newcommand\litsup[1]{Citation \supercite{#1}}
\bibliography{../literature}


\begin{document}

\tableofcontents
\newpage

\section{Introduction}
This is a short example on how to use the bibstyler-Template. \\
Now we referer to the following literature:

\begin{itemize}
 \item{\bf Trends in Theoretical Computer Science \lit{boerger_trends83}}
 \item{\bf Science of Computer Programming \lit{SCP}}
 \item{\bf An Explanation of Splaying \lit{Subramanian:1994:ES}}
 \item{\bf Computer Science: The Nature of a Good and Accreditable Computer Science Program \lit{Martin96}}
\end{itemize}

And the same example with $\backslash$litsup:

\begin{itemize}
 \item{\bf Trends in Theoretical Computer Science \litsup{boerger_trends83}}
 \item{\bf Science of Computer Programming \litsup{SCP}}
 \item{\bf An Explanation of Splaying \litsup{Subramanian:1994:ES}}
 \item{\bf Computer Science: The Nature of a Good and Accreditable Computer Science Program \litsup{Martin96}}
\end{itemize}

\newpage

% Literaturverzeichnis
\newpage
\phantomsection
\addcontentsline{toc}{section}{Literature}
\printbibliography[maxnames=99]

\end{document}

